\title{CS 341 Project Proposal: \\
Predicting Ridesharing Demand}
\author{
        Ramon Iglesias \\
            \and
        Federico Rossi\\
        	\and
        Kevin Wang
}
\date{\today}

\documentclass[10pt]{article}

\begin{document}
\maketitle

\section{Problem Description}

Personal mobility in the form of privately owned automobiles contributes to increasing levels of traffic congestion, pollution, and under-utilization of vehicles (on average 5\% in the US \cite{DN:15}) -- clearly unsustainable trends for the future. The pressing need to reverse these trends has spurred the creation of cost competitive, on-demand personal mobility solutions such as car-sharing (e.g. Car2Go, ZipCar) and ride-sharing (e.g. Uber, Lyft).
However, without proper fleet management, car-sharing and ride-sharing systems lead to vehicle imbalances: vehicles aggregate in some areas while becoming depleted in others, due to the asymmetry between trip origins and destinations \cite{RZ-MP:15_MODa}. This issue has been addressed in a variety of ways in the literature. The works in \cite{MN-SZ-SB-MJR:15}, \cite{BB-KGZ-NG:15}, and \cite{FA-DDP-AR:14} investigate using paid drivers to move vehicles between car-sharing stations where cars are parked, while \cite{SB-RJ-CR:15} studies the merits of dynamic pricing for incentivizing drivers to move to underserved areas. Moreover, \cite{RZ-MP:15_MODa,RI-FR-RZ-MP:16,RZ-FR-MP:16,RZ-FR-MP:16a} tackle the problem under the assumption of \emph{autonomous} vehicles, and cast the problem from the perspective of optimal control.

Nonetheless, these approaches are reactive, that is, they react to current state conditions rather predicted future conditions, e.g. future customer demand. We believe that robust and accurate customer demand prediction models would enable a new family of car- and ride-sharing rebalancing problems. Thus, we propose the development of large-scale predictive models that provide mobility supply and demand prediction at different time horizons.

\section{Data}

We will leverage the dataset provided by DiDi. Assuming the data is similar in content (but not in size) to the one provided for last year's DiDi Research Challenge \footnote{http://research.xiaojukeji.com/competition/detail.action?competitionId=DiTech2016} and with the preliminary information provided by Professor Leskovec, we know that there are the following tables:

\begin{itemize}
  \item {\emph{Order Info}} -- Trip requests associating passengers with drivers, origins, destinations, timestamps and prices. 
  \item {\emph{Points of Interest}} -- Attributes of the different districts (which serve as origin and destination regions).
  \item {\emph{Weather Info}} -- Timestamped weather information including temperature, pollution, and whether it is sunny, raining, etc.
  \item {\emph{Traffic Jam Info}} -- Level of traffic congestion in each district based throughout time. Traffic is binned into a finite set of categories.
\end{itemize}

A cursory exploration of the DiDi Research Challenge data, shows that the Order Info contains requests where the passenger was not matched, thus enabling us to leverage such requests as unmet demand. The pricing information will also help us model price sensitivities for both drivers and passengers, as well as proxies for cost of moving between districts (which is relevant when setting dynamic prices).

\section{Methodology}\label{methods}


\section{Evaluation}\label{evaluation}


\section{Plan}\label{plan}


\section{About Us}\label{about}

\subsection*{Ramon Iglesias}
Ramon Iglesias is a PhD student at the Autonomous Systems Laboratory. His research focuses on optimal control of Autonomous Mobility-on-Demand systems. Prior to resuming his PhD, Ramon worked at SunPower as a Financial Software Engineer where he helped build the current pricing and analytics engine which powers the bulk of SunPower's residential sales. He recently took CS246, and has previous machine learning experience in the form courses and research projects.

\subsection*{Federico Rossi}

\subsection*{Kevin Wang}
\bibliographystyle{abbrv}
\bibliography{main}

\end{document}
This is never printed